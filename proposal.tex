\documentclass{article}

\usepackage{microtype}

\usepackage{amsmath}
\usepackage{mathtools}

\title{CS225 Spring 2018---Final Project Proposal}
\author{
  Aaron Longchamp \\ \small{\texttt{gitlab.uvm.edu:@alongcha}}
  \and Connor Allan \\ \small{\texttt{gitlab.uvm.edu@crallan}}
}
\date{\today}

\begin{document}
\maketitle

\section*{Project: Exceptions for Type Checkers}

We propose to implement a semantics for System F extended with error
types (error, try, raise exception and handle exception).

\paragraph{Base Language}

We will work with simply typed lambda calculus with booleans, natural numbers,
let-binding, and products as the base language.

\paragraph{Extended Language}

We will extend this language with the features of System F and error
types. This consists of five new types:
\begin{enumerate}
\item A type variable, written $X$
\item A type error, written $error$
\item A type try, written $try$
\item A raise exception denoted raise, written $raise X$
\item A handle exception denoted try with, written $try X with X$
\end{enumerate}
and five new terms:
\begin{enumerate}
\item Type abstraction, written $\Lambda X.e$
\item Type error, written $\Gamma \vdash error : X$
\item Type try, written $\Gamma \vdash try x_1 with x_2 X$
\item Type raise exception, $\Gamma \vdash raise x_1 :X$
\item Type handle exception, $\Gamma \vdash try x_1 with x_2 : X$
\end{enumerate}


\paragraph{Applications}

Error handling types have applications to all ways of programming. Errors 
occur when the system runs into the issue of not being able to preform a 
task or function correctly -because of some calculation that would involve 
a division by zero or an arithmetic overflow, a lookup key is missing from 
a dictionary, an array index went out of bounds, a file could not be found 
or opened, some disastrous event occurred such as the system running out 
of memory or the user killing the process, etc (Pierce, 171). The purpose
of an exception handler is to prevent such issues from happening and stopping
a system wide crash. 

\paragraph{Timeline and Milestones}

By the checkpoint we hope to have completed:
\begin{enumerate}
\item A prototype implementation of the small-step semantics
\item A suite of test-cases for the small-step semantics and well-typed relation
\item One program encoded in Ocaml which demonstrates the application of exception
      types
\end{enumerate}

\noindent
By the final project draft we hope to have completed:
\begin{enumerate}
\item The full implementation of small-step semantics and type checking
\item A fully comprehensive test suite, with all tests passing
\item The program in Ocaml running through both the semantics and type
  checker implementation
\item A draft writeup that explains the implementation of exception types
\item A presentation with at least 5 slides as the starting point for our
  in-class presentation
\end{enumerate}

\noindent
By the final project submissions we hope to have completed:
\begin{enumerate}
\item The final writeup and presentation
\item Any remaining implementation work that was missing in the final project
  draft
\end{enumerate}

\end{document}
